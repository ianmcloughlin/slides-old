\documentclass{beamer}

% Most of the preamble is in here.
\usepackage{iansslides}

% Title, subtitle, author, and date.
\title{New slides}
\subtitle{Hello, World!}
\author{ian.mcloughlin@gmit.ie}
\date{\today}

% Start of the document.
\begin{document}

% Title frame.
\section{Turing machines}


% A frame with Tikz.
\begin{frame}{Visualisation}
  \begin{adjustbox}{max totalsize={0.9\textwidth}{\textheight-3\baselineskip}, keepaspectratio, center}
    \begin{tikzpicture}
      \draw (0,0) coordinate (origin);
      \draw (5,5) coordinate (center);
      \draw (origin) -- +(1,1);
      \draw (2,2) -- (4,4); 
      \draw[red] (45:6) -- (45:7);
      \draw[gray] (0,0) grid (10,10);
      \draw (origin) .. controls (10,4) and (10,6) .. (0,10);
      \draw (center) node[draw,shape=circle] (Hello) {$H^el^lo$};
      \foreach \i in {0,1,2,3,4}
      {
        \path (center) -- +(90+\i*360/5:4) coordinate  (P\i);
        \path[draw,fill=gmitblue] (P\i) circle (1);
      }
      \begin{scope}[shift={(1,5)}]
        \draw[smooth,domain=0:6.5] plot function{sin(2*x)*exp(-x/4)};
      \end{scope}
    \end{tikzpicture}
  \end{adjustbox}
\end{frame}

% A frame with Tikz.
% \begin{frame}{Turing machine}
%   \begin{adjustbox}{max totalsize={0.9\textwidth}{\textheight-3\baselineskip}, keepaspectratio, center}
%     \begin{tikzpicture}
%       \begin{scope}[start chain]
%       \foreach \i in {0,1,2,3,4}
%       {
%         \node[on chain] (cell\i) {\i};
%       }
%     \end{tikzpicture}
%   \end{adjustbox}
% \end{frame}

\end{document}