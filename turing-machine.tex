\documentclass{beamer}

% Most of the preamble is in here.
\usepackage{iansslides}

\begin{document}

% Title frame.
\section{Turing machines}

\begin{frame}{Example 1: Turing machine}
  \begin{adjustbox}{max totalsize={0.9\textwidth}{\textheight-3\baselineskip}, keepaspectratio, center}
    \begin{tikzpicture}
      % Tape: left dotted lines.
      \draw (1,0) -- (8,0);
      \draw (1,1) -- (8,1);
      % Tape: top and bottom solid lines.
      \draw[densely dotted] (0.5,0) -- (1,0);
      \draw[densely dotted] (0.5,1) -- (1,1);
      % Tape: right dotted lines.
      \draw[densely dotted] (8,0) -- (8.5,0);
      \draw[densely dotted] (8,1) -- (8.5,1);
      % Tape: leftmost solid vertical.
      \draw (1,0) -- (1,1);
      % Tape: cells
      \foreach \x [count = \i] in {0,1,0,0,1,0,0}
      {
        % Tape: cell right solid vertical.
        \draw (\i+1,0) -- (\i+1,1);
        % Tape: cell text.
        \draw (\i+0.5,0.5) node {$\x$};
      }
      % State with arrow above a cell.
      \draw[gmitred,latex-] (2.5,1.1) -- (2.5,1.5) node[draw,shape=rectangle,anchor=south] {$q_0$};
    \end{tikzpicture}
  \end{adjustbox}
  \vspace{4mm}
  \begin{table}
    \begin{tabular}{ccccc}
      State & Read & Write & Move & Next \\
      \midrule
      $q_0$ & $\square$ & $\square$ & L & $q_a$ \\
      $q_0$ & $0$       & $1$       & R & $q_0$ \\
      $q_0$ & $1$       & $0$       & R & $q_0$ \\
    \end{tabular}
  \end{table}
\end{frame}


\end{document}